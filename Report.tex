\documentclass[11pt]{article}

% --- Layout & quality ---
\usepackage[margin=1in]{geometry}     % standard 1-inch margins
\usepackage{microtype}                % better spacing/kerning
\usepackage{parskip}                  % space between paragraphs, no indents
\usepackage{float}

% --- Math, tables, graphics, links ---
\usepackage{amsmath, amssymb}
\usepackage{booktabs}                 % pro-looking tables
\usepackage{siunitx}                  % nice numbers/units in tables
\sisetup{round-mode=places,round-precision=2,table-number-alignment=center}
\usepackage{graphicx}                 % \includegraphics
\usepackage[hidelinks]{hyperref}      % clickable refs without boxes

% --- Title (edit freely) ---
\title{Labor Force Participation and Human Capital Attainment: A Survey of the United States 2010-2019}

\author{John Fullerton}
\date{October 12, 2025}

\begin{document}
\maketitle

% ===========================
% ABSTRACT (write ~3-5 sentences)
% Goal: What you studied, how you did it (data + method), key quantitative finding, and why it matters.
% Keep to ~90--130 words.
% ===========================
\begin{abstract}
% TODO: Replace this placeholder with your own summary.
I analyze the relationship between the monthly labor force participation rate (LFPR) and human capital attainment in each state from 2010-2019, later aggregating the data for nationwide insight. Using monthly, per-state labor series from FRED and education data from the US Census Bureau, I conduct an OLS and two-way fixed-effect regression to identify correlations between and within states over time. I find that there is a statistically significant correlation between the share of the population obtaining at least a bachelor's degree and the labor force participation rate among all states from 2008 to 2019. However, there is a notable lack of significant correlation between increases in education rates and LFPR within each state when accounting for all other time-invariant factors, including economic mobility, fiscal policies, and geography.
\end{abstract}

% ===========================
% 1. INTRODUCTION (6–8 sentences)
% What is the research question, why it matters (economic intuition/policy), and your one-sentence main result.
% Tip: End with a roadmap sentence (what comes next).
% ===========================
\section{Introduction}
% TODO: Write 6–8 sentences here.
One of the main statistics of study in the field of development economics is the labor force participation rate. Although moderately subjected to cyclical influences, LFPR encompasses the amount of labor resources available in an economy for the production of goods and services, which is generally the product of longer-term structural characteristics of an economy. These include the strength of institutions\footnote{Daron Acemoglu and James A. Robinson, *Why Nations Fail: The Origins of Power, Prosperity, and Poverty* (New York: Crown Business, 2012), 45.} and labor incentives, including Social Security and employer-provided pensions\footnote{Monthly Labor Review, September 2016, Steven Hipple}. In his Nobel Prize winning work, Acemoglu refined his definition of institutions to 'inclusive institutions' that broadly distribute intellectual, innovative, and political power to the people. In this report, I aim to clarify the relationship between LFPR and enrollment in Universities, which play a major role as inclusive institutions in distributing human capital. With a better understanding of this correlation, I hope to equip myself for future study in development economics with the mission of improving access to inclusive institutions around the world.
% ===========================
% 2. DATA & METHODS (short and concrete)
% - Sources and time span
% - Key variables: how you defined education and LFPR (units!)
% - Model in one line (with fixed effects if used)
% ===========================
\newpage
\section{Data and Methods}

This analysis uses a state-level panel dataset covering all 50 U.S. states and Washington, D.C., from January 2010 to December 2019 (N = 510 state-year observations). The dependent variable is the annualized average Labor Force Participation Rate (LFPR) for each state. The monthly LFPR data were sourced from the Federal Reserve Economic Data (FRED) "CIVPART" series and annualized to match the frequency of the explanatory variable, while leveraging as many data points as possible for statistical robustness.

The key independent variable is the bachelor's\_share, defined as the portion of the US population 25 years of age and above while holding at least a bachelor's degree. This data was sourced from the U.S. Census Bureau's American Community Survey (ACS) 1-year estimates.

To examine the relationship between these variables, I first deploy an Ordinary Least Squares (OLS) model:

$$\text{LFPR}_{it} = \beta_0 + \beta_1 (\text{bachelors\_share})_{it} + \epsilon_{it}$$

In this model, $i$ indexes the state and $t$ indexes the year. This model, however, certainly suffers from omitted variable bias, as it fails to account for unobserved, time-invariant differences between states including working culture and overall economic prosperity (invariant in this range). There are likely other time-variant factors that are omitted, and are outside the scope of this project, including age demographics, and business development.

I therefore select a Two-Way Fixed Effects (TWFE) model, which aims to address such issues by ignoring state and year fixed effects. This is calculated by subtracting the average LFPR rates from each entry, leaving only the variation caused by changes in education to be modeled.

$$\text{LFPR}_{it} = \beta_1 (\text{bachelors\_share})_{it} + \alpha_i + \gamma_t + \epsilon_{it}$$

Here, $\alpha_i$ (alpha) represents the fixed effects of the state, which absorb all of the time-invariant characteristics from each state $i$. The term $\gamma_t$ (gamma) represents the fixed effects of the year, which captures any variation at the national-level common to all states in year $t$. This TWFE model isolates the effect of bachelor's share by only using within-state variation (i.e., how a state's LFPR changes when its own education share changes), providing a more credible estimate of the relationship.


% ===========================
% 3. RESULTS (keep tight; aim for one figure + one table)
% - Figure: show the core relationship (scatter or map)
% - Table: a tiny regression table (coef + SE)
% - Interpret magnitudes, not just signs
% ===========================
\section{Results}

% Upload your figure (PNG/PDF) via Overleaf and replace the filename below.
% Use width=0.9\linewidth to keep it readable on 1–2 pages.
\begin{figure}[H]
    \centering
    % TODO: Replace 'edu_lfpr_scatter.png' with your actual uploaded filename (no spaces).
    \includegraphics[width=0.9\linewidth]{LFPR Education Regression Plot.pdf}
    \caption{Education and LFPR by state-year. Each point is a state-year; line shows fitted values.}
    \label{fig:scatter}
\end{figure}

\subsection*{Regression evidence}
% Option A: quick inline table (edit numbers).

\begin{table}[H]
\centering
\caption{Effect of Education Share on Labor Force Participation Rate}
\label{tab:main_regression}
\begin{tabular}{lcc}
\toprule
& (1) & (2) \\
& OLS (Pooled) & Two-way Fixed Effects \\
\midrule
\textit{Variables} & & \\
bachelors\_share & 0.306*** & -0.085* \\
 & (0.024) & (0.050) \\
 & & \\
(Intercept) & 54.909*** & \\
 & (0.752) & \\
\midrule
\textit{Fixed Effects} & & \\
State Fixed Effects & No & Yes \\
Year Fixed Effects & No & Yes \\
\midrule
\textit{Model Statistics} & & \\
Observations & 510 & 510 \\
R$^{2}$ & 0.236 & 0.975 \\
R$^{2}$ (Within) & -- & 0.006 \\
\bottomrule
\multicolumn{3}{l}{\textit{Note:} Standard errors in parentheses.} \\
\multicolumn{3}{l}{* p<0.1, ** p<0.05, *** p<0.01} \\
\end{tabular}
\end{table}

% Interpretation: 2–3 sentences. Translate coef into an effect size that a policymaker understands.
\newpage
The regression results in Table \ref{tab:main_regression} highlight the prominent difference between the two models. The pooled OLS model in Column (1) finds a positive and statistically significant coefficient (0.306, p$<$0.01). This result, which can be seen in Figure \ref{fig:scatter}, suggests that a 1 unit  increase in a state's bachelor's share is associated with a 0.306\% increase in LFPR. However, the TWFE model in Column (2), which is expected to be more accurate, tells a very different story. After controlling for all time-invariant state characteristics ($\alpha_i$) and national-level trends ($\gamma_t$), the relationship disappears. The coefficient becomes small and negative (-0.085) and is only weakly significant (p$<$0.1).This reversal implies that the positive OLS result was likely influenced by variables that were omitted in this report. This implies that unobserved factors that are stable over time in a given state, including economic strength and structure are likely correlated with both higher education shares and higher labor force participation. The OLS model naively attributed this correlation to education. 

% ===========================
% 4. DISCUSSION (tight paragraph)
% What this *does* and *does not* show; key threats to interpretation; 1–2 robustness notes.
% ===========================
\section{Discussion}

The key finding of this analysis is that the simple, positive correlation between higher education and labor force participation appears to be limited, driven by unobserved, time-invariant differences between states rather than a causal effect within them.

This report does not aspire to contribute to or critique existing literature in the space.  With that said, I hope this report may serve as a resource for aspiring economic researchers as it did myself. The formatting and structure of this report was motivated by professional publications in the space, and employed tools that will only become more valuable in the progression of one's academic career. 



% ===========================
% REFERENCES (manual, so no BibLaTeX/biber needed)
% Add 2–4 key sources: data + any methods references you cited.
% ===========================
\section*{References}
\begin{flushleft}
% TODO: Replace with your actual sources (Agency/Title/Years/URL or DOI).
\textbf{\textbf{Data.}} Federal Reserve Economic Data (FRED). \emph{Civilian Labor Force Participation Rate by State (Seasonally Adjusted, 'LBSSA' Series)}. 2010--2019 (monthly, aggregated to annual). \url{https://fred.stlouisfed.org/}
\vspace{0.5em} % Adds a little space

\textbf{Data.} U.S. Census Bureau. \emph{American Community Survey (ACS) 1-Year Estimates, Table S1501: Educational Attainment}. 2010--2019. \url{https://data.census.gov/}

\textbf{Methods.} Imai, Kosuke, and In Song Kim. "On the Use of Two-Way Fixed Effects Regression Models for Causal Inference with Panel Data." Working Paper, 2020. \url{https://web.mit.edu/insong/www/pdf/FEmatch-twoway.pdf}
\end{flushleft}

% ===========================
% APPENDIX (optional, keep super short to stay within 1–2 pages)
% ===========================
% \section*{Appendix (optional)}
% Brief variable definitions; extra robustness note; or one more figure/table if space remains.

\end{document}
